\documentclass[aspectratio=169]{beamer}
%% For 4:3 aspect ratio:
%% \documentclass{beamer}
\usepackage{../git-course}

\title[git-course]{Setting up Git and \gh\ for the first time}
\author{Chris Grandin \& Andrew Edwards}
\date{\today}

\begin{document}

\frame[plain]{
\titlepage
}

\frame{\frametitle{Getting Started...}
  Before you start using Git, you need to set up your computer to use it.
  This is a one-time setup and once it is done, you can easily create
  new projects or join others in collaboration.
}

\frame{\frametitle{Get a \gh\ account}
  \bigskip
  \bi
    \item Sign up for \gh: \url{https://github.com}
    \item Have a photo (headshot) handy, and attach it to your profile. Makes it easy for collaborators to identify you.
  \ei
}

\frame{\frametitle{Windows only: Make sure you have .NET 4.5}
  \bi
  \item The Windows \gh\ desktop application requires Microsoft .NET 4.5. If
    you have a .NET version less than 4.5 then upgrade it. Check your version
    of .NET here:\\
    \url{https://github.com/downloads/shanselman/SmallestDotNet/CheckForDotNet45.exe}
  \item If you need to upgrade it 4.5:\\
    \url{http://go.microsoft.com/fwlink/p/?LinkId=310158}
  \ei
}

\frame{\frametitle{Install the Git application on your machine}
  \bi
    \item Windows:\\
      \url{http://windows.github.com}
    \item MAC: Type \emph{git} on the command line. You will either see that it
      is installed, or if you have OSX 10.9 or higher it will prompt you to
      install it. If your OSX is less than 10.9, or for any other reason, you
      can install git from here:\\
      \url{https://git-scm.com/download/mac}
  \ei
}

\frame{\frametitle{Configure the Git application}
  \bi
    \item Windows:
      \bn
        \item Create the directory \lstinline[escapechar="\\"]{C:\\github}. If
          you decide to use a different directory, make sure there are no spaces
          or special characters anywhere in the full path.
        \item Open the GitHub Desktop Application. Press the gear icon and
          choose \emph{Options}.
        \item In \emph{Configure Git}, fill in your name and the
          email address you used for your GitHub account.
        \item Change your \emph{Clone Path} to \lstinline[escapechar="\\"]{C:\\github}
          (or whatever directory you created above).
        \item Make sure that for \emph{Default Shell}, \emph{PowerShell} is
          checked.
        \item Click \emph{Save} and close the application.
        \item Check the configuration by opening the \gs, (not the \gh\
          application). The directory that it starts in should be
          \lstinline[escapechar="\\"]{C:\\github} (or whatever directory you created
          above).
      \en
      \item MAC:
        \bn
          \item Create the directory \lstinline[escapechar="\~"]{\~/github}.
          \item Enjoy a beverage.
        \en
  \ei
}

\frame{\frametitle{Install the difftool}
  The difftool will be used to examine differences between different
  versions of files and also to simplify merging of branches and
  collaborator's code. There are many programs that can be used but
  for consistency we will use Diffmerge.\\
  \bigskip

  For both Windows and MAC, Install Diffmerge:\\
  \bigskip
  \url{https://sourcegear.com/diffmerge/downloads.php}\\
  \bigskip
  The configuration for directing git to use Diffmerge will be done in a
  later step.
}

\frame{\frametitle{Cloning the git-course repository}
  For these instructions, replace \lstinline{GITHUB-USER} with your
  \gh\ account name.
  \bi
    \item On the \gh\ webpage, sign into your account and navigate to:\\
      \url{https://github.com/cgrandin/git-course}
    \item Press the \emph{Fork} button. This will give you a copy of the
      git-course repository.
    \item Navigate to your new forked repository:\\
      \lstinline{https://github.com/GITHUB-USER/git-course}
    \item Copy the URL for your repository.
    \item Windows: Open the \gs\ and run the following command to clone the
      repository, you can paste the URL into the \gs\ by pressing the right
      mouse button:\\
      \lstinline{git clone https://github.com/GITHUB-USER/git-course git-course}
    \item MAC: Open terminal and change to the github directory:\\
      \lstinline{cd ~/github}\\
      then run the clone command:\\
      \lstinline{git clone https://github.com/GITHUB-USER/git-course git-course}
  \ei
}

\frame{\frametitle{Copy the \emph{.gitconfig} file}
  \bi
    \item Git uses a configuration file for your account info, name to use
      when committing, aliases for commands, and other things. We will start
      with one I use and modify it.
    \item Open up the \emph{contents} sub-directory in the \emph{git-course}
      directory and copy the file \emph{.gitconfig}
    \item For Windows, copy this file (overwrite the existing file) to:\\
      \bigskip
      \lstinline[escapechar="\\"]{C:\\Users\\YOUR-COMPUTER-USER-NAME\\.gitconfig}\\
      \bigskip
      where YOUR-COMPUTER-USER-NAME is your user name on your computer, not
      your \gh\ account name.
      \bigskip
    \item For MAC, copy this file (overwrite the existing file) to:\\
      \bigskip
      \lstinline[escapechar="\~"]{\~/.gitconfig}
      \bigskip
    \ei
}

\frame{\frametitle{Edit the \emph{.gitconfig} file}
  \bi
    \item Use your favourite editor to edit the file. Don't edit the one in the
      \emph{git-course/contents} directory.
    \item Change the [user] settings.
    \item Change the [difftool] and [diffmerge] directories so they point to
      the location where you have DiffMerge.
    \item For Windows the location should be:\\
      \bigskip
      \lstinline[escapechar="\\"]{C:\\Program Files\\SourceGear\\Common\\DiffMerge\\sgdm.exe}
      \bigskip
    \item For MAC the location should be:\\
      \bigskip
      \lstinline{/usr/local/bin/diffmerge}
      \bigskip
    \ei
}

\frame{\frametitle{MAC only: Make your output pretty}
  On the MAC, change to the \lstinline[escapechar="\~"]{\~/github} directory and run the
  following command:\\
  \bigskip
  \lstinline{git config --global color.ui.auto}\\
  \bigskip
  This will make your git output colored in a similar way to the Windows
  powershell version.
}

\frame{\frametitle{If you want to build these slides..}
  If you want to build the slideshows we have prepared from source, you need to
  have \LaTeX\ installed. I suggest trying, because using \emph{beamer} is great!
  \bi
    \item On Windows, install Mik\TeX:\\
      \bigskip
      \url{https://miktex.org/download}\\
      \bigskip
      Build each pdf by entering the beamer directory and then
      the subdirectories inside. Each will have a \emph{build-pdf.bat} file which
      you can double-click.
    \item On MAC, install Mac\TeX:\\
      \bigskip
      \url{http://www.tug.org/mactex/}\\
      \bigskip
      Run the following in your shell in the directory for the given slideshow:\\
      \lstinline{pdflatex TEX\_FILE\_NAME}\\
      where TEX\_FILE\_NAME is the name of the TEX file, e.g.:
      \lstinline{git-course-setup.tex}.\\
      Run it two or three times to ensure page numbers are accurate.
  \ei
  \bigskip
}

\frame{\frametitle{You're ready...}
  That's it! You should have everything set up to create new repositories, fork
  collaborator's repositories, commit changes, push your commits, create
  branches, and view differences. During the course you'll understand what all that means.\\
  \bigskip
  To get a sneak peek at what is to come, take a look at the git-course
  repository on \gh:\\
  \bigskip
  \url{https://github.com/cgrandin/git-course}\\
  \bigskip
  In particular, the network graph which shows all our commits from the
  beginning:\\
  \bigskip
  \url{https://github.com/cgrandin/git-course/network}\\
  \bigskip
  The network graph is an invaluable tool in keeping track of what is going
  on amongst you and you collaborators. We'll get into it more in the course.

  See you soon!
}

\end{document}
