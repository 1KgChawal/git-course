\documentclass[aspectratio=169]{beamer}
%% For 4:3 aspect ratio:
%% \documentclass{beamer}
\mode<presentation>

\title[git-course]{Setting up Git and \gh\ for use with this course}
\author{Chris Grandin \& Andy Edwards}
\date{\today}

\usetheme[compress]{Singapore}
\usecolortheme{seahorse}
\usefonttheme{default}

\setbeamertemplate{headline}[default]
\setbeamercolor{mycolor}{fg=white,bg=structure!30}
\setbeamertemplate{navigation symbols}{}
\addtobeamertemplate{footline}{
    \leavevmode
    \hbox{
      \begin{beamercolorbox}[wd=\paperwidth,
          ht=2.75ex,
          dp=.5ex,
          right,
          rightskip=1em]{mycolor}
        \usebeamercolor[fg]{navigation symbols}
        \insertslidenavigationsymbol
        \insertframenavigationsymbol
        \insertsubsectionnavigationsymbol
        \insertsectionnavigationsymbol
        \insertdocnavigationsymbol
        \insertbackfindforwardnavigationsymbol
    \end{beamercolorbox}
    }
    \vskip0.5pt
}{}

\addtobeamertemplate{navigation symbols}{}{
    \usebeamerfont{footline}
    \usebeamercolor[fg]{footline}
    \hspace{1em}
    \insertframenumber/\inserttotalframenumber
}

\usepackage{url}
\usepackage{hyperref}
\hypersetup{colorlinks=true,
            linkcolor=blue,
            urlcolor=blue}
\usepackage{graphicx}
\usepackage{pgf}

\newcommand{\bc}{\begin{center}}
\newcommand{\ec}{\end{center}}
\newcommand{\bn}{\begin{enumerate}}
\newcommand{\en}{\end{enumerate}}
\newcommand{\bi}{\begin{itemize}}
\newcommand{\ei}{\end{itemize}}
\newcommand{\gh}{GitHub}
\newcommand{\gs}{Git shell}

\begin{document}

\frame[plain]{
\titlepage
}

\frame{\frametitle{Getting Started...}
  Before we start using Git, you need to set up your computer to use it.
  This is a one-time setup and once it is done, you can easily create
  new projects or join others in collaboration.
}

\frame{\frametitle{Get a \gh\ account}
  Remember which email address you used to sign up, as you'll need to
  enter it in the configuration later.
  \bigskip
  \bi
    \item \url{https://github.com}
  \ei
}

\frame{\frametitle{Windows only: Make sure you have .NET 4.5}
  \bi
  \item The Windows \gh\ application requires Microsoft .NET 4.5. If you have
    a .NET version less than 4.5 then upgrade it. Check your version of .NET
    here:\\
    \url{https://github.com/downloads/shanselman/SmallestDotNet/CheckForDotNet45.exe}
  \item If you need to upgrade it to 4.5:\\
    \url{http://go.microsoft.com/fwlink/p/?LinkId=310158}
  \ei
}

\frame{\frametitle{Install the Git application on your machine}
  \bi
    \item Windows:\\
      \url{http://windows.github.com}
    \item MAC: Type \emph{git} on the command line. You will either see that it
      is installed, or if you have OSX 10.9 or higher it will prompt you to
      install it. If your OSX is less than 10.9, or for any pther reason, you
      can install git from here:\\
      \url{https://git-scm.com/download/mac}
  \ei
}

\frame{\frametitle{Configure the Git application}
  \bi
    \item Windows:
      \bn
        \item Create the directory C:\textbackslash github.
        \item Open the GitHub Desktop Application. Press the gear icon and
          choose \emph{Options}.
        \item In \emph{Configure Git}, fill in your name and the
          email address you used for your GitHub account.
        \item Change your \emph{Clone Path} to C:\textbackslash github.
        \item Make sure that for \emph{Default Shell}, \emph{PowerShell} is
          checked.
        \item Click \emph{Save} and close the application.
        \item Check the configuration by opening the \gs, (not the \gh\
          application). The directory that it starts in should be
          C:\textbackslash github.
      \en
      \item MAC:
        \bn
          \item Create the directory \textasciitilde/github.
          \item Enjoy a beverage.
        \en
  \ei
}

\frame{\frametitle{Install the difftool}
  The difftool will be used to examine differences between different
  versions of files and also to simplify merging of branches and
  collaborator's code. There are many programs that can be used but
  for consistency we will use Diffmerge.\\
  \bigskip

  For both Windows and MAC, Install Diffmerge:\\
  \bigskip
  \url{https://sourcegear.com/diffmerge/downloads.php}\\
  \bigskip
  The configuration for directing git to use Diffmerge will be shown later.
}

\frame{\frametitle{Cloning the git-course repository}
  \bi
    \item On the \gh\ webpage, sign into your account and navigate to:\\
      \url{https://github.com/cgrandin/git-course}
    \item Press the \emph{Fork} button. This will give you a copy of the
      git-course repository.
    \item Navigate to your new forked repository (replace GITHUB-USER with
      your \gh\ account name):\\
      \url{https://github.com/GITHUB-USER/git-course}
    \item Copy the URL for your repository
    \item Windows: Open the \gs\ and run the following command to clone the
      repository, you can paste the URL into the \gs\ by pressing the right
      mouse button:\\
      git clone \url{https://github.com/GITHUB-USER/git-course} git-course
    \item MAC: Open terminal and change to the github directory, then run
      the clone command:\\
      cd \textasciitilde/github\\
      git clone \url{https://github.com/GITHUB-USER/git-course} git-course
  \ei
}

\frame{\frametitle{Copy the \emph{.gitconfig} file}
  \bi
    \item Git uses a configuration file for your account info, name to use
      when committing, aliases for commands, and other things. We will start
      with one I use and modify it.
    \item Open up the \emph{contents} sub-directory in the \emph{git-course}
      directory and copy the file \emph{.gitignore}
    \item For Windows, copy this file to:\\
      \bigskip
      C:\textbackslash Users\textbackslash YOUR-COMPUTER-USER-NAME\textbackslash
      .gitconfig\\
      \bigskip
      where YOUR-COMPUTER-USER-NAME is your user name on your computer, not
      your \gh\ account name.
      \bigskip
    \item For MAC, copy this file to:\\
      \bigskip
      \textasciitilde/.gitconfig
      \bigskip
    \ei
}

\frame{\frametitle{Edit the \emph{.gitconfig} file}
  \bi
    \item Use your favourite editor to edit the file. Don't edit the one in the
      \emph{git-course/contents} directory.
    \item Change the [user] settings.
    \item Change the [difftool] and [diffmerge] directories so they point to
      the location where you have DiffMerge.
    \item For Windows the location should be:\\
      \bigskip
      C:\textbackslash Program Files\textbackslash SourceGear\textbackslash
      Common\textbackslash DiffMerge\textbackslash sgdm.exe
      \bigskip
    \item For MAC the location should be:\\
      \bigskip
      /usr/local/bin/diffmerge
      \bigskip
    \ei
}

\frame{\frametitle{You're ready...}
  That's it! You should have everything set up to create new repositories, fork
  collaborator's repositories, commit changes, push your commits, create
  branches, and view differences.\\
  \bigskip
  To get a sneak peek at what is to come, take a look at the git-course
  repository on \gh:\\
  \bigskip
  \url{https://github.com/cgrandin/git-course}\\
  \bigskip
  In particular, the network graph which shows all our commits from the
  beginning:\\
  \bigskip
  \url{https://github.com/cgrandin/git-course/network}\\
  \bigskip
  The network graph is an invaluable tool in keeping track of what is going
  on amongst you and you collaborators. We'll get into it more in the course.

  See you soon!
}


%% Maybe this will be a different presentation...
%% \section{Why use Git?}
%% \frame{\frametitle{Why use git?}
%%   \bi
%%     \item Collaboration
%%     \item Version control
%%     \item Organization
%%   \ei
%% }

%% \section{Git Workflow}
%% \frame{\frametitle{Git Workflow}
%%   Decide what kind of a project you are working on. Is it something that
%%   people contribute small parts of sporadically, or is it something that
%%   requires collaboration between people working closely on a project?

%%   \bi
%%     \item A project where collaborators work closely together
%%       \bi
%%         \item Fork
%%         \item Merge other people's remote repositories using \gs
%%       \ei
%%     \item A project where people contribute small parts to a \emph{master}
%%       \bi
%%         \item Don't Fork
%%         \item Pull request on the \gh\ site
%%       \ei
%%   \ei

%%   We will focus on the first method, as we are mostly working on assessments
%%   or other smallish projects closely with collaborators.
%% }

%% \frame{\frametitle{Creating a new repository}
%%   \bi
%%     \item Once you have a \gh\ account, sign in, click on the
%%       \emph{Repositories} tab, and press the \emph{New} button.
%%     \item Give your repository a name. I like to use small letters and dashes
%%       between words.
%%     \item Check \emph{Initialize this repository with a README}.
%%     \item Leave \emph{Add .gitignore} and \emph{Add a license}
%%       set to \emph{None}.
%%     \item Click \emph{Create repository}.
%%   \ei
%% }

%% \frame{\frametitle{Cloning your new repository}
%%   \bi
%%     \item Once you have created a repository on \gh, copy the URL
%%       of the repository. e.g. \url{https://github.com/cgrandin/git-course}
%%     \item Open the \gs\ and run the following command to clone your repository,
%%       you can paste the URL into the \gs\ by pressing the right mouse button:
%%     \item \emph{git clone URL git-course}
%%   \ei
%% }


%% \frame{\frametitle{Storing your credentials}
%%   git config --global credential.helper wincred
%% }

%% \frame{\frametitle{A Great Git page}
%%   \url{https://www.git-tower.com/learn/git/ebook/en/command-line}
%% }

\end{document}
