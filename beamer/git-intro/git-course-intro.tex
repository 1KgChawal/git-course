\documentclass[aspectratio=169]{beamer}
%% For 4:3 aspect ratio:
%% \documentclass{beamer}
\usepackage{../git-course}

\title[git-course]{Setting up Git and \gh\ for use with this course}
\author{Chris Grandin \& Andy Edwards}
\date{\today}


\begin{document}

\frame[plain]{
\titlepage
}

\section{Why use Git?}
\frame{\frametitle{Why use git?}
  \bi
    \item Collaboration
    \item Version control
    \item Organization
  \ei
}

\section{Git Workflow}
\frame{\frametitle{Git Workflow}
  Decide what kind of a project you are working on. Is it something that
  people contribute small parts of sporadically, or is it something that
  requires collaboration between people working closely on a project?

  \bi
    \item A project where collaborators work closely together
      \bi
        \item Fork
        \item Merge other people's remote repositories using \gs
      \ei
    \item A project where people contribute small parts to a \emph{master}
      \bi
        \item Don't Fork
        \item Pull request on the \gh\ site
      \ei
  \ei

  We will focus on the first method, as we are mostly working on assessments
  or other smallish projects closely with collaborators.
}

\frame{\frametitle{Creating a new repository}
  \bi
    \item Once you have a \gh\ account, sign in, click on the
      \emph{Repositories} tab, and press the \emph{New} button.
    \item Give your repository a name. I like to use small letters and dashes
      between words.
    \item Check \emph{Initialize this repository with a README}.
    \item Leave \emph{Add .gitignore} and \emph{Add a license}
      set to \emph{None}.
    \item Click \emph{Create repository}.
  \ei
}

\frame{\frametitle{Cloning your new repository}
  \bi
    \item Once you have created a repository on \gh, copy the URL
      of the repository. e.g. \url{https://github.com/cgrandin/git-course}
    \item Open the \gs\ and run the following command to clone your repository,
      you can paste the URL into the \gs\ by pressing the right mouse button:
    \item \emph{git clone URL git-course}
  \ei
}

\frame{\frametitle{Storing your credentials}
  git config --global credential.helper wincred
}

\frame{\frametitle{Add collaborators}
  If your repository is public, anyone can post an issue. If you want someone to
  be able to:
  \bigskip
  \bi
    \item Set assignees to issues
    \item Receive notifications when you reply to their issue
  \ei
  \bigskip
  You need to add them as a collaborator. Select the \emph{Settings} tab and then
  the \emph{Collaborators} button. Type their \gh\ username in and send them an
  invitation to collaborate.
}

\frame{\frametitle{A Great Git page}
  \url{https://www.git-tower.com/learn/git/ebook/en/command-line}
}

\end{document}
